% This file should be replaced with your file with an thesis content.
%=========================================================================
% Authors: Michal Bidlo, Bohuslav Křena, Jaroslav Dytrych, Petr Veigend and Adam Herout 2019

\hyphenation{OpenStack Swift OpenIO SDS}

\chapter{Introduction}

% general introduction for a topic:
%- cloud computing popularity
%- most popular service: cloud storage and its types
%- about object storage
%- users interest of what's going in theirs storage / event activities
%- TODO: maybe remove talk about types of cloud storages? solution: talk about need for monitoring and users need to monitor/receive information regarding their storage
In the current world, cloud computing became the most popular way of delivering different types of services throught Internet. One of the most popular cloud service is cloud storage, which allows users to store data in remote locations mentained by third party. Based on how cloud storage manages data, cloud storage can be divided into 3 types: Block storage, File storage and Object storage. Object storage manages data as objects, each object typically includes data itself and some additional informations stored in objects metadata. Since data are stored in remote locations, to which users don't have direct and complete access, some users or external services might want to receive informations about certain events (for example change of content) in storages where their data are located.

% importance of thesis in this field
%- react to events - possibly react to 'bad' events
%- allows users to have better picture what is going on in their storage
Importance of this thesis is to provide event informations to users in OpenIO SDS and OpenStack Swift, which will allow user to react on those events and possibly prevent/detect unwanted actions. Providing event notifications will allow users to have better picture on what is going on in their storage and impove monitoring in these object storages.

%past advances in this field
There was two attempts\cite{swiftPatch1}\cite{swiftPatch2} to solve this issue within OpenStack Swift which were not officially accepted and their solution is outdated. Currently there is no official solution for publishing event notification in OpenStack Swift nor OpenIO Software-Defined Storage (hereinafter SDS).

%why am I interested in this, why I choose this topic
%- perosnally i used object storage
%- can se myself using this solution in future as well as millions of other users
%- impact on big amount of users
%- contibuting to open source projects
My interest in this topic steems from its possible impact to extensive amount of users that OpenStack Swift and OpenIO have. Contibuting to open source projects is something that I always wanted to be part of. Possiblity to improve user experience in OpenStack Swift and OpenIO SDS and allow those storages to be even more competative against comercial storages (Amazon, Google, ...) is another reason why I choose this topic.

%goals of thesis
The goal of this thesis is to create program/middleware which wil publish event notification to user specified destination. One of supported destination will be Beanstalk queue, but program will allow to easily add other types of destinations (for example Kafka) using predefined interface. Proposed program will allow user to specify, using objects metadata (such as name prefix/sufix and object size) and type of event, which event notification should be published. Program will be able to run within OpenStack Swift as well as OpenIO SDS. This thesis will strive to find such solution that could be officially accpeted as part of OpenStack Swift and OpenIO SDS.

%structure of thesis
Structure of thesis - TODO - there will be probably change in chpaters structure

\chapter{Background}

This chapter introduces Object storage and its core concepts along with the underlying technologies. For sufficient understanding its important to explain how Software-defined storage manages data and what types of events can occur inside. The last part of this chapter describes concept of event notifications, why are they important and current interfaces for publishing event notifications to users.

\section{Object storage}
    %introduction/concept
    Object storage, also known as \textit{object-based storage} (OBS), is type of storage that handles data as objects, instead of hierarchical methods used in file systems\cite{objectBasedStorage}. Object stores are designed at handling data as whole objects, making them ideal solution for any unchanging data. Data in object stores are changed by replacing objects or files and therefore object stores are prefered mechanicsm for storing such files\cite{networkStorage}.

    %key koncepts
    \subsection{Key concepts}
    Key concepts of object storage are\cite{ibmObjectStorage}:
    \begin{itemize}
        \item{Objects - An object typically consist of user data and metadata uploaded to objected storage.}
        \item{Containers/Buckets - represents logical abstraction that is used to provide a data container in object storage. An object with same name in two different containers represents two different objects. This concept is used to segragate data using bucket ownership and a combination of public and secret keys bound to object store accounts which allows users and application to manipulate with data that are authorized for specific type of manipulation (read/write/update).}
        \item{Metadata - Additional information about data, create and last modified date, size, hash,...}
        \item{Access Control Lists(ACLs) - used as primary security construct in object storage, stored in account or bucket level and allows owners to grant permissions for certaint operations based on UUID, email, ...}
        \item{Object Data protection - two main schemes of data protections in object storage are \textbf{Replication} and \textbf{Erasure Coding}.

        Replication is method to ensure data resilience. Data are copied into multiple locations/disks/partitions, in case of failure, data are used from secondary copy, either to recreate original copy or as main copy.

        Erasure coding is process throught the data is separated into fragments. Then fragments are expanded and encoded with redundant pieces and stored across different storage devices. Erasure coding adds redundancy and allows object storage to tolerate failures.}
    \end{itemize}


    \subsection{Object data}
    %types of stored informations
    With object storage techniques, each object contains\cite{ibmObjectStorage}:
    \begin{itemize}
      \item{Data - user specified data that needs to be stored in persistent storage. It can be binary data, text file, image, etc..}
      \item{Metadata - Extra data describing objects data and can be divided into two types: Device-managed metadata is additional information maintained by storage device and used as part of object management in physical storage.\cite{objectBasedStorage}. Second type is Custom metadata, where user can store any additional information in key and value pairs. In object storage metadata are stored together with the object.}
      \item{A universally unique identifier (UUID) - This ID, created using hashing process based on object name and some other additional informations, is assigned to each object in a Object storage. Using ID object storage systems are cappable of tell a part objects from one another and ID is used to extract data in system without knowing their physical location/drive and offset.}
    \end{itemize}

    \subsection{Access to object storage}
    %access to object storage
    %-resfull interface
    %-advantages of this approach
    %-popular intefaces
    Obeject storage services provide RESTful interface \cite{cloudObjectStorage} over HTTP protocol for objects store and access. This approach allows users to create, read, delete, update or even query objects anytime and anywhere simply by referencing UUID (or using certain attributes for querying), usually with proper authentication process. Most popular interfaces for comunicating with object storages are Amazon S3 (Simple Storage Service) API and OpenStack Swift API.

    %comparison to other storage models
    \subsection{Pros and cons of object storage}
    Pros:
    \begin{itemize}
        \item{Cappable of handling large amount of unstructured data}
        \item{Reduced TCO and cheap COTS - Object storage is designed to utilize cheap COTS(Commercial off-the-shelf) components, as result Total Cost of Ownership(TCO) is lower then owning home-made Network-Attached Storages(NAS)\cite{networkStorage}.}
        \item{Unlimited scalability - Since object storages are build on distributed systems, they scale very well compared to traditional storages where they often have upper limit.\cite{openstackObjectStorage}}
        \item{Wide-open metadata - allowing user to store custom metadata and posibility of createing metadata-driven policies such as compression and tiering.}
    \end{itemize}
    Cons:
    \begin{itemize}
        \item{No in-place update - object must be manipulated as whole unit}
        \item{No locking mechanicsm - object storage does not manage object-level locking and it is up to applications to solve concurrent PUT/GET.}
        \item{Slower - this makes object storages poor choice for applications that need rapid and frequent acess to data.}
    \end{itemize}


    \begin{figure}[hbt]
        \centering
        \includegraphics[width=0.7\textwidth]{obrazky-figures/placeholder.pdf}
        \caption{TODO figure}
    \end{figure}
\section{Software-defined storage}
    \textcolor{gray}{\Blindtext}
    \begin{figure}[hbt]
        \centering
        \includegraphics[width=0.7\textwidth]{obrazky-figures/placeholder.pdf}
        \caption{TODO figure}
    \end{figure}

\section{Event notifications}
    \textcolor{gray}{\Blindtext}
\subsection{Amazon S3 event notifications}
    \textcolor{gray}{\Blindtext}
    \begin{figure}[hbt]
        \centering
        \includegraphics[width=0.7\textwidth]{obrazky-figures/placeholder.pdf}
        \caption{TODO figure}
    \end{figure}
\subsection{Oracle event notifications}
    \textcolor{gray}{\Blindtext}
    \begin{figure}[hbt]
        \centering
        \includegraphics[width=0.7\textwidth]{obrazky-figures/placeholder.pdf}
        \caption{TODO figure}
    \end{figure}

\chapter{Object Storages}
    \textcolor{gray}{\Blindtext}
\section{OpenIO SDS}
    \textcolor{gray}{\Blindtext}
    \begin{figure}[hbt]
        \centering
        \includegraphics[width=0.7\textwidth]{obrazky-figures/placeholder.pdf}
        \caption{TODO figure}
    \end{figure}
\section{OpenStack Swift}
    \textcolor{gray}{\Blindtext}
    \begin{figure}[hbt]
        \centering
        \includegraphics[width=0.7\textwidth]{obrazky-figures/placeholder.pdf}
        \caption{TODO figure}
    \end{figure}
\section{MinIO}
    \textcolor{gray}{\Blindtext}
    \begin{figure}[hbt]
        \centering
        \includegraphics[width=0.7\textwidth]{obrazky-figures/placeholder.pdf}
        \caption{TODO figure}
    \end{figure}

\chapter{Solution draft}
    \textcolor{gray}{\Blindtext}
\section{Current state}
    \textcolor{gray}{\Blindtext}
\section{Middleware for OpenStack Swift and OpenIO SDS}
    \textcolor{gray}{\Blindtext}
    \begin{figure}[hbt]
        \centering
        \includegraphics[width=0.7\textwidth]{obrazky-figures/placeholder.pdf}
        \caption{TODO figure}
    \end{figure}
    \textcolor{gray}{\Blindtext}
    \begin{figure}[hbt]
        \centering
        \includegraphics[width=0.7\textwidth]{obrazky-figures/placeholder.pdf}
        \caption{TODO figure}
    \end{figure}
\section{Adapter for MinIO}
    \textcolor{gray}{\Blindtext}
    \begin{figure}[hbt]
        \centering
        \includegraphics[width=0.7\textwidth]{obrazky-figures/placeholder.pdf}
        \caption{TODO figure}
    \end{figure}

\chapter{Implementation, experiments and assessment}

\chapter{Conclusion}

%=========================================================================
